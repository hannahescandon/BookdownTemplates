% Options for packages loaded elsewhere
\PassOptionsToPackage{unicode}{hyperref}
\PassOptionsToPackage{hyphens}{url}
%
\documentclass[
]{article}
\usepackage{amsmath,amssymb}
\usepackage{lmodern}
\usepackage{ifxetex,ifluatex}
\ifnum 0\ifxetex 1\fi\ifluatex 1\fi=0 % if pdftex
  \usepackage[T1]{fontenc}
  \usepackage[utf8]{inputenc}
  \usepackage{textcomp} % provide euro and other symbols
\else % if luatex or xetex
  \usepackage{unicode-math}
  \defaultfontfeatures{Scale=MatchLowercase}
  \defaultfontfeatures[\rmfamily]{Ligatures=TeX,Scale=1}
\fi
% Use upquote if available, for straight quotes in verbatim environments
\IfFileExists{upquote.sty}{\usepackage{upquote}}{}
\IfFileExists{microtype.sty}{% use microtype if available
  \usepackage[]{microtype}
  \UseMicrotypeSet[protrusion]{basicmath} % disable protrusion for tt fonts
}{}
\makeatletter
\@ifundefined{KOMAClassName}{% if non-KOMA class
  \IfFileExists{parskip.sty}{%
    \usepackage{parskip}
  }{% else
    \setlength{\parindent}{0pt}
    \setlength{\parskip}{6pt plus 2pt minus 1pt}}
}{% if KOMA class
  \KOMAoptions{parskip=half}}
\makeatother
\usepackage{xcolor}
\IfFileExists{xurl.sty}{\usepackage{xurl}}{} % add URL line breaks if available
\IfFileExists{bookmark.sty}{\usepackage{bookmark}}{\usepackage{hyperref}}
\hypersetup{
  pdftitle={Assignment},
  pdfauthor={Hannah Escandon - escandon.h@northeastern.edu},
  hidelinks,
  pdfcreator={LaTeX via pandoc}}
\urlstyle{same} % disable monospaced font for URLs
\usepackage[margin=1in]{geometry}
\usepackage{color}
\usepackage{fancyvrb}
\newcommand{\VerbBar}{|}
\newcommand{\VERB}{\Verb[commandchars=\\\{\}]}
\DefineVerbatimEnvironment{Highlighting}{Verbatim}{commandchars=\\\{\}}
% Add ',fontsize=\small' for more characters per line
\usepackage{framed}
\definecolor{shadecolor}{RGB}{248,248,248}
\newenvironment{Shaded}{\begin{snugshade}}{\end{snugshade}}
\newcommand{\AlertTok}[1]{\textcolor[rgb]{0.94,0.16,0.16}{#1}}
\newcommand{\AnnotationTok}[1]{\textcolor[rgb]{0.56,0.35,0.01}{\textbf{\textit{#1}}}}
\newcommand{\AttributeTok}[1]{\textcolor[rgb]{0.77,0.63,0.00}{#1}}
\newcommand{\BaseNTok}[1]{\textcolor[rgb]{0.00,0.00,0.81}{#1}}
\newcommand{\BuiltInTok}[1]{#1}
\newcommand{\CharTok}[1]{\textcolor[rgb]{0.31,0.60,0.02}{#1}}
\newcommand{\CommentTok}[1]{\textcolor[rgb]{0.56,0.35,0.01}{\textit{#1}}}
\newcommand{\CommentVarTok}[1]{\textcolor[rgb]{0.56,0.35,0.01}{\textbf{\textit{#1}}}}
\newcommand{\ConstantTok}[1]{\textcolor[rgb]{0.00,0.00,0.00}{#1}}
\newcommand{\ControlFlowTok}[1]{\textcolor[rgb]{0.13,0.29,0.53}{\textbf{#1}}}
\newcommand{\DataTypeTok}[1]{\textcolor[rgb]{0.13,0.29,0.53}{#1}}
\newcommand{\DecValTok}[1]{\textcolor[rgb]{0.00,0.00,0.81}{#1}}
\newcommand{\DocumentationTok}[1]{\textcolor[rgb]{0.56,0.35,0.01}{\textbf{\textit{#1}}}}
\newcommand{\ErrorTok}[1]{\textcolor[rgb]{0.64,0.00,0.00}{\textbf{#1}}}
\newcommand{\ExtensionTok}[1]{#1}
\newcommand{\FloatTok}[1]{\textcolor[rgb]{0.00,0.00,0.81}{#1}}
\newcommand{\FunctionTok}[1]{\textcolor[rgb]{0.00,0.00,0.00}{#1}}
\newcommand{\ImportTok}[1]{#1}
\newcommand{\InformationTok}[1]{\textcolor[rgb]{0.56,0.35,0.01}{\textbf{\textit{#1}}}}
\newcommand{\KeywordTok}[1]{\textcolor[rgb]{0.13,0.29,0.53}{\textbf{#1}}}
\newcommand{\NormalTok}[1]{#1}
\newcommand{\OperatorTok}[1]{\textcolor[rgb]{0.81,0.36,0.00}{\textbf{#1}}}
\newcommand{\OtherTok}[1]{\textcolor[rgb]{0.56,0.35,0.01}{#1}}
\newcommand{\PreprocessorTok}[1]{\textcolor[rgb]{0.56,0.35,0.01}{\textit{#1}}}
\newcommand{\RegionMarkerTok}[1]{#1}
\newcommand{\SpecialCharTok}[1]{\textcolor[rgb]{0.00,0.00,0.00}{#1}}
\newcommand{\SpecialStringTok}[1]{\textcolor[rgb]{0.31,0.60,0.02}{#1}}
\newcommand{\StringTok}[1]{\textcolor[rgb]{0.31,0.60,0.02}{#1}}
\newcommand{\VariableTok}[1]{\textcolor[rgb]{0.00,0.00,0.00}{#1}}
\newcommand{\VerbatimStringTok}[1]{\textcolor[rgb]{0.31,0.60,0.02}{#1}}
\newcommand{\WarningTok}[1]{\textcolor[rgb]{0.56,0.35,0.01}{\textbf{\textit{#1}}}}
\usepackage{graphicx}
\makeatletter
\def\maxwidth{\ifdim\Gin@nat@width>\linewidth\linewidth\else\Gin@nat@width\fi}
\def\maxheight{\ifdim\Gin@nat@height>\textheight\textheight\else\Gin@nat@height\fi}
\makeatother
% Scale images if necessary, so that they will not overflow the page
% margins by default, and it is still possible to overwrite the defaults
% using explicit options in \includegraphics[width, height, ...]{}
\setkeys{Gin}{width=\maxwidth,height=\maxheight,keepaspectratio}
% Set default figure placement to htbp
\makeatletter
\def\fps@figure{htbp}
\makeatother
\setlength{\emergencystretch}{3em} % prevent overfull lines
\providecommand{\tightlist}{%
  \setlength{\itemsep}{0pt}\setlength{\parskip}{0pt}}
\setcounter{secnumdepth}{-\maxdimen} % remove section numbering
\usepackage{titling}
\pretitle{\begin{center} \includegraphics[width=2in,height=2in]{res/NEU_logo21.png}\LARGE\\}
\posttitle{\end{center}}
\ifluatex
  \usepackage{selnolig}  % disable illegal ligatures
\fi
\newlength{\cslhangindent}
\setlength{\cslhangindent}{1.5em}
\newlength{\csllabelwidth}
\setlength{\csllabelwidth}{3em}
\newenvironment{CSLReferences}[2] % #1 hanging-ident, #2 entry spacing
 {% don't indent paragraphs
  \setlength{\parindent}{0pt}
  % turn on hanging indent if param 1 is 1
  \ifodd #1 \everypar{\setlength{\hangindent}{\cslhangindent}}\ignorespaces\fi
  % set entry spacing
  \ifnum #2 > 0
  \setlength{\parskip}{#2\baselineskip}
  \fi
 }%
 {}
\usepackage{calc}
\newcommand{\CSLBlock}[1]{#1\hfill\break}
\newcommand{\CSLLeftMargin}[1]{\parbox[t]{\csllabelwidth}{#1}}
\newcommand{\CSLRightInline}[1]{\parbox[t]{\linewidth - \csllabelwidth}{#1}\break}
\newcommand{\CSLIndent}[1]{\hspace{\cslhangindent}#1}

\title{Assignment}
\author{Hannah Escandon -
\href{mailto:escandon.h@northeastern.edu}{\nolinkurl{escandon.h@northeastern.edu}}}
\date{27 December, 2021}

\begin{document}
\maketitle

{
\setcounter{tocdepth}{2}
\tableofcontents
}
\hypertarget{start-here}{%
\subsection{Start here}\label{start-here}}

This is text that I wrote here - just follow these examples.

\hypertarget{after-you-write}{%
\subsection{After you write}\label{after-you-write}}

you can `knit' this document into an HTML document - like a website or
you can create a PDF or a word document.

Look at at the submenu items and you will see `Knit' next to a ball of
yarn. Click that. Select - Knit to HTML or Knit to PDF or Knit of Doc.

You can add images here as well

\includegraphics[width=3.58333in,height=\textheight]{https://www.rstudio.com/wp-content/uploads/2018/10/RStudio-Logo.png}

\newpage

\begin{Shaded}
\begin{Highlighting}[]
\FunctionTok{library}\NormalTok{(reticulate)}
\end{Highlighting}
\end{Shaded}

\begin{verbatim}
## Warning: package 'reticulate' was built under R version 4.1.2
\end{verbatim}

\begin{Shaded}
\begin{Highlighting}[]
\FunctionTok{library}\NormalTok{(tidyverse)}
\end{Highlighting}
\end{Shaded}

\begin{verbatim}
## -- Attaching packages --------------------------------------- tidyverse 1.3.1 --
\end{verbatim}

\begin{verbatim}
## v ggplot2 3.3.5     v purrr   0.3.4
## v tibble  3.1.6     v dplyr   1.0.7
## v tidyr   1.1.4     v stringr 1.4.0
## v readr   2.1.1     v forcats 0.5.1
\end{verbatim}

\begin{verbatim}
## Warning: package 'tibble' was built under R version 4.1.2
\end{verbatim}

\begin{verbatim}
## Warning: package 'tidyr' was built under R version 4.1.2
\end{verbatim}

\begin{verbatim}
## Warning: package 'readr' was built under R version 4.1.2
\end{verbatim}

\begin{verbatim}
## -- Conflicts ------------------------------------------ tidyverse_conflicts() --
## x dplyr::filter() masks stats::filter()
## x dplyr::lag()    masks stats::lag()
\end{verbatim}

\begin{Shaded}
\begin{Highlighting}[]
\FunctionTok{library}\NormalTok{(tidytext)}
\end{Highlighting}
\end{Shaded}

\begin{verbatim}
## Warning: package 'tidytext' was built under R version 4.1.2
\end{verbatim}

\begin{Shaded}
\begin{Highlighting}[]
\FunctionTok{library}\NormalTok{(plotly)}
\end{Highlighting}
\end{Shaded}

\begin{verbatim}
## Warning: package 'plotly' was built under R version 4.1.2
\end{verbatim}

\begin{verbatim}
## 
## Attaching package: 'plotly'
\end{verbatim}

\begin{verbatim}
## The following object is masked from 'package:ggplot2':
## 
##     last_plot
\end{verbatim}

\begin{verbatim}
## The following object is masked from 'package:stats':
## 
##     filter
\end{verbatim}

\begin{verbatim}
## The following object is masked from 'package:graphics':
## 
##     layout
\end{verbatim}

\hypertarget{introduction}{%
\subsection{Introduction}\label{introduction}}

This template expands on the minimal example, and includes a logo and
custom CSS.

USE THE VISUAL EDITOR

Bajuk (n.d.)

\hypertarget{analysis}{%
\subsection{Analysis}\label{analysis}}

\hypertarget{load-file-from-url}{%
\subsection{Load file from url}\label{load-file-from-url}}

\begin{Shaded}
\begin{Highlighting}[]
\NormalTok{seedURL }\OtherTok{\textless{}{-}} \StringTok{"https://archive.ics.uci.edu/ml/machine{-}learning{-}databases/00236/seeds\_dataset.txt"}
\NormalTok{seedColumns }\OtherTok{\textless{}{-}} \FunctionTok{c}\NormalTok{(}\StringTok{"area"}\NormalTok{,}\StringTok{"perimeter"}\NormalTok{,}\StringTok{"compactness"}\NormalTok{,}\StringTok{"length"}\NormalTok{,}\StringTok{"width"}\NormalTok{,}\StringTok{"asym"}\NormalTok{,}\StringTok{"groove"}\NormalTok{,}\StringTok{"cluster"}\NormalTok{)}
\NormalTok{seedDF }\OtherTok{\textless{}{-}}  \FunctionTok{read\_table}\NormalTok{(seedURL,}\AttributeTok{col\_names =}\NormalTok{ seedColumns)}
\end{Highlighting}
\end{Shaded}

\begin{verbatim}
## 
## -- Column specification --------------------------------------------------------
## cols(
##   area = col_double(),
##   perimeter = col_double(),
##   compactness = col_double(),
##   length = col_double(),
##   width = col_double(),
##   asym = col_double(),
##   groove = col_double(),
##   cluster = col_double()
## )
\end{verbatim}

\begin{Shaded}
\begin{Highlighting}[]
\CommentTok{\#seedDF \textless{}{-}  read\_table("./seeds\_dataset.txt",col\_names = seedColumns)}
\FunctionTok{glimpse}\NormalTok{(seedDF)}
\end{Highlighting}
\end{Shaded}

\begin{verbatim}
## Rows: 210
## Columns: 8
## $ area        <dbl> 15.26, 14.88, 14.29, 13.84, 16.14, 14.38, 14.69, 14.11, 16~
## $ perimeter   <dbl> 14.84, 14.57, 14.09, 13.94, 14.99, 14.21, 14.49, 14.10, 15~
## $ compactness <dbl> 0.8710, 0.8811, 0.9050, 0.8955, 0.9034, 0.8951, 0.8799, 0.~
## $ length      <dbl> 5.763, 5.554, 5.291, 5.324, 5.658, 5.386, 5.563, 5.420, 6.~
## $ width       <dbl> 3.312, 3.333, 3.337, 3.379, 3.562, 3.312, 3.259, 3.302, 3.~
## $ asym        <dbl> 2.2210, 1.0180, 2.6990, 2.2590, 1.3550, 2.4620, 3.5860, 2.~
## $ groove      <dbl> 5.220, 4.956, 4.825, 4.805, 5.175, 4.956, 5.219, 5.000, 5.~
## $ cluster     <dbl> 1, 1, 1, 1, 1, 1, 1, 1, 1, 1, 1, 1, 1, 1, 1, 1, 1, 1, 1, 1~
\end{verbatim}

\begin{Shaded}
\begin{Highlighting}[]
\CommentTok{\# Exclude column 8}
\NormalTok{seedDF7 }\OtherTok{\textless{}{-}}\NormalTok{ seedDF }\SpecialCharTok{\%\textgreater{}\%} \FunctionTok{select}\NormalTok{(}\SpecialCharTok{{-}}\NormalTok{cluster)}
\end{Highlighting}
\end{Shaded}

\hypertarget{cluster}{%
\subsection{Cluster}\label{cluster}}

\begin{Shaded}
\begin{Highlighting}[]
\NormalTok{featureString }\OtherTok{=} \FunctionTok{c}\NormalTok{(}\StringTok{"area"}\NormalTok{,}\StringTok{"perimeter"}\NormalTok{)}
\NormalTok{c }\OtherTok{=} \DecValTok{2}

\NormalTok{kmresult }\OtherTok{\textless{}{-}}\NormalTok{ seedDF }\SpecialCharTok{\%\textgreater{}\%} 
  \FunctionTok{select}\NormalTok{(featureString) }\SpecialCharTok{\%\textgreater{}\%}
  \FunctionTok{kmeans}\NormalTok{(}\AttributeTok{centers =}\NormalTok{ c,}\AttributeTok{nstart =} \DecValTok{10}\NormalTok{)}
\end{Highlighting}
\end{Shaded}

\begin{verbatim}
## Note: Using an external vector in selections is ambiguous.
## i Use `all_of(featureString)` instead of `featureString` to silence this message.
## i See <https://tidyselect.r-lib.org/reference/faq-external-vector.html>.
## This message is displayed once per session.
\end{verbatim}

\begin{Shaded}
\begin{Highlighting}[]
\CommentTok{\# Get the cluster result and add to the dataframe}
\NormalTok{seedDF7}\SpecialCharTok{$}\NormalTok{cluster }\OtherTok{\textless{}{-}} \FunctionTok{as.character}\NormalTok{(kmresult}\SpecialCharTok{$}\NormalTok{cluster)}




\NormalTok{p3 }\OtherTok{\textless{}{-}} \FunctionTok{ggplot}\NormalTok{() }\SpecialCharTok{+}  \FunctionTok{geom\_point}\NormalTok{(}\AttributeTok{data =}\NormalTok{ seedDF7,}
                  \AttributeTok{mapping =} \FunctionTok{aes\_}\NormalTok{(}
                           \AttributeTok{x =} \FunctionTok{sym}\NormalTok{(featureString[}\DecValTok{1}\NormalTok{]), }
                           \AttributeTok{y =} \FunctionTok{sym}\NormalTok{(featureString[}\DecValTok{2}\NormalTok{]), }
                           \AttributeTok{colour =}\NormalTok{ seedDF7}\SpecialCharTok{$}\NormalTok{cluster)) }\SpecialCharTok{+}
                  \FunctionTok{geom\_point}\NormalTok{(}\AttributeTok{mapping =} \FunctionTok{aes\_string}\NormalTok{(}
                    \AttributeTok{x =}\NormalTok{ kmresult}\SpecialCharTok{$}\NormalTok{centers[, featureString[}\DecValTok{1}\NormalTok{]], }
                    \AttributeTok{y =}\NormalTok{ kmresult}\SpecialCharTok{$}\NormalTok{centers[, featureString[}\DecValTok{2}\NormalTok{]]),}
                    \AttributeTok{color =} \StringTok{"red"}\NormalTok{, }\AttributeTok{size =} \DecValTok{4}\NormalTok{) }\SpecialCharTok{+}
                \FunctionTok{labs}\NormalTok{(}
                  \AttributeTok{title =} \StringTok{"KMeans{-} Total sum of squares"}\NormalTok{,}
                  \AttributeTok{subtitle =}\NormalTok{ kmresult}\SpecialCharTok{$}\NormalTok{tot.withinss}
\NormalTok{  )}
\end{Highlighting}
\end{Shaded}

\hypertarget{print-plotly-only-if-html}{%
\subsubsection{Print plotly only if
html}\label{print-plotly-only-if-html}}

\begin{Shaded}
\begin{Highlighting}[]
\CommentTok{\# Only print if html output}
\ControlFlowTok{if}\NormalTok{(knitr}\SpecialCharTok{::}\FunctionTok{is\_html\_output}\NormalTok{())\{}
  \FunctionTok{ggplotly}\NormalTok{(p3)}
\NormalTok{\} }\ControlFlowTok{else}\NormalTok{ \{}
\NormalTok{  p3}
\NormalTok{\}}
\end{Highlighting}
\end{Shaded}

\includegraphics{skeleton_files/figure-latex/unnamed-chunk-5-1.pdf}

\begin{Shaded}
\begin{Highlighting}[]
\NormalTok{kmresult}\SpecialCharTok{$}\NormalTok{tot.withinss}
\end{Highlighting}
\end{Shaded}

\begin{verbatim}
## [1] 510.0478
\end{verbatim}

\hypertarget{principal-component-analysis}{%
\subsubsection{Principal Component
Analysis}\label{principal-component-analysis}}

\begin{Shaded}
\begin{Highlighting}[]
\CommentTok{\# Which features should I keep?  All? Some?}
\CommentTok{\#seedPCA \textless{}{-} prcomp(seedDF7)}
\CommentTok{\#summary(seedPCA)}
\end{Highlighting}
\end{Shaded}

\hypertarget{refs}{}
\begin{CSLReferences}{1}{0}
\leavevmode\hypertarget{ref-bajuk}{}%
Bajuk, Lou. n.d. {``Debunking the Myths of r Vs. Python.''}
\url{https://blog.rstudio.com/2021/06/15/debunking-the-myths-of-r-vs-python/}.

\end{CSLReferences}

\end{document}
